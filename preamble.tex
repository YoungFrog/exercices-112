% \def\pgfsysdriver{pgfsys-dvi.def}
\usepackage{pstricks-add}
\usepackage{amsfonts,amssymb,amsmath,amsthm,latexsym,algorithmic,babel}
\usepackage[T1]{fontenc}
\usepackage[utf8]{inputenc}
\usepackage{multicol}
\usepackage{xstring}
\usepackage{lmodern}
\usepackage{needspace}
\usepackage{venndiagram}
\usepackage{numprint}
\usepackage{enumitem}
\usepackage{pgf,tikz,comment,xfrac}
\newtheorem*{correction}{Correction}
\newtheorem*{example}{Exemple}
\usepackage{systeme}
\usepackage{siunitx}
% this is for backward compat with my own systeme environment.
\let\ubpsysteme\systeme
\renewenvironment{systeme}{\left\{\begin{array}{r@{\,=\,}l}}{\end{array}\right.}
\newenvironment{inlinesysteme}{$\displaystyle\systeme}{\endsysteme$}
\newcommand{\requires}[1]{}%{\message{^^JREQUIRES: #1^^J}}
\usetikzlibrary{snakes,arrows}
\frenchbsetup{IndentFirst=false,ItemLabels=\textbullet} % must be removed if french is not used.

\usepackage[babel=true,kerning=true]{microtype}
% pour régler un conflit entre babel et tikz


\usepackage{graphicx}
\graphicspath{{graphics/}}
\makeatletter
\def\input@path{{graphics/}}
\makeatother

%\textheight253mm\voffset-24mm\textwidth170mm\hoffset-9mm
\title{}
\author{}
\date{}
\pagestyle{headings}
%%%%%%%%%%%%%%%%%%%% MACROS
\long\global\def\C#1\F{{}}
\newcommand{\E}{\mathbb E}
\newcommand{\N}{\mathbb N{}}
\newcommand{\Q}{\mathbb Q}
\newcommand{\R}{\mathbb R{}}
\newcommand{\Z}{\mathbb Z{}}

\newcommand{\CCC}{\mathcal{C}{}}

%%%%%%%%%%%%% Notation française pour les fonction trig. et hyperb. et les fonctions inverses. J. Leroy, janvier 1993.

\DeclareMathOperator{\arccosec}{arccos\acute{e}c}
\DeclareMathOperator{\rot}{rot}
\def\arccoseca{\arccosec}
\DeclareMathOperator{\arccotg}{arccotg}
\DeclareMathOperator{\arcsec}{arcs\acute{e}c}
\def\arcseca{\arcsec}

\def\arctg{\arctan}
\DeclareMathOperator{\Id}{Id}
\DeclareMathOperator{\argch}{argch}
\DeclareMathOperator{\argcosecah}{argcos\acute{e}ch}
\DeclareMathOperator{\argcoth}{argcoth}
\DeclareMathOperator{\argsecah}{args\acute{e}ch}
\DeclareMathOperator{\argsh}{argsh}
\DeclareMathOperator{\argth}{argth}

\DeclareMathOperator{\ch}{ch}
\DeclareMathOperator{\coseca}{cos\acute{e}c}
\DeclareMathOperator{\cosecah}{cos\acute{e}ch}
\DeclareMathOperator{\cotg}{cotg}
%\DeclareMathOperator{\coth}{coth}
\DeclareMathOperator{\seca}{s\acute{e}c}
\DeclareMathOperator{\secah}{s\acute{e}ch}
\DeclareMathOperator{\sh}{sh}
\DeclareMathOperator{\tg}{tg}

%%%%%%%%%%% jpd sept. 2007

\def\CCC{\mathcal{C}{}}
%\newcommand{\arccosec}{\mathrm{arccos\acute ec}}
%\newcommand{\arccoseca}{\arccosec}
%\newcommand{\arccotg}{\mathrm{arccotg}}
%\newcommand{\arcsec}{\mathrm{arcs\acute ec}}
%\newcommand{\arcseca}{\arcsec}
%\newcommand{\arctg}{\mathrm{arctg}}

%\newcommand{\argch}{\mathrm{argch}}
%\newcommand{\argcoth}{\mathrm{argcoth}}
%\newcommand{\argsh}{\mathrm{argsh}}
%\newcommand{\argth}{\mathrm{argth}}

%\newcommand{\ch}{\mathrm{ch}}
%\newcommand{\coseca}{\mathrm{cos\acute ec}}
%\newcommand{\cotg}{\mathrm{cotg}}
%\newcommand{\seca}{\mathrm{s\acute ec}}
%\newcommand{\sh}{\mathrm{sh}}
%\newcommand{\tg}{\mathrm{tg}}
%\renewcommand{\th}{\mathrm{th}}
%%%%%%%%%%%%

\newcommand{\BB}[1]{\mathbb{#1}}
\newcommand{\cqfd}{\nolinebreak\hfill\rule{2mm}{2mm}\vspace{2mm}\par}
\newcommand{\F}{\overrightarrow}
\newcommand{\bt}{\bigtriangleup}
\newcommand{\ari}[3]{\stackrel{\scriptstyle #1#2#3}{\hphantom{#1}<\,
\hphantom{#3}}}
\newcommand{\ars}[3]{\stackrel{\scriptstyle #1#2#3}{\hphantom{#1}>\,
\hphantom{#3}}}
\newcommand{\lig}[3]{\lim_{\ari{#1}{#2}{#3}}}
\newcommand{\lid}[3]{\lim_{\ars{#1}{#2}{#3}}}

\def\Im{\mathrm{Im}{}}
\def\Ker{\mathrm{Ker}{}}
\def\vect{\mathrm{vect}\,}

\newcommand{\dint}[1]{\begin{array}[t]{c}
\displaystyle{\int\!\!\!\int}\\#1\;\;\end{array}}
\newcommand{\tint}[1]{\begin{array}[t]{c}
\displaystyle{\int\!\!\!\int\!\!\!\int}\\#1\;\;\end{array}}
\newcommand{\dintf}[1]{\begin{array}[t]{c}
\displaystyle{\int\!\!\!\!\!\int}
\mbox{\hspace{-13.7pt}}
\bigcirc\;\\#1\;\;\end{array}}

\newcommand{\fdi}{\arraycolsep0.1pt \renewcommand{\arraystretch}{0.5}
\begin{array}[t]{c}
\to\\
<
\end{array}
\arraycolsep5pt \renewcommand{\arraystretch}{1}}

%Définition d'une flèche à droite avec un signe > en dessous.
\newcommand{\fds}{\arraycolsep0.1pt \renewcommand{\arraystretch}{0.5}
\begin{array}[t]{c}\to\\
>
\end{array}
\arraycolsep5pt \renewcommand{\arraystretch}{1}}

%\renewcommand{\det}{\mathrm{d\acute et}\,}
% macro a ameliorer en partant du \det original
\newcommand{\tr}{\mathrm{tr}\,}

\newcommand{\Sym}{\mathrm{Sym}{}}

\newcommand{\addmod}{\mathbin{\mathrm{mod}}}%???

%%%%%%%%%%%%%%%%%%%%%%%%%%%%%%
\newcounter{numexo}[chapter]
\setcounter{numexo}{0}
\newenvironment{exo}{
  \par\vspace{1mm}
  \refstepcounter{numexo}%
  \noindent{\bfseries \thenumexo}.~~}{}
\def\thenumexo{\thechapter.\arabic{numexo}}
\newcounter{numque}
\setcounter{numque}{0}        
\newcommand{\que}{\stepcounter{numque}
\vskip2.5mm\noindent{\bfseries\arabic{numque}}.~}  
%       debut de question

\def\sque#1{\vskip1mm\noindent\hbox to 5mm{\hfil#1})~}  

%%%%%%%%%%%%%%%%%%%%%%%%%%%%%%
                 
%\usepackage{theorem}
\newtheorem{thm}{Th\'eor\`eme}[chapter]
\newtheorem{prop}[thm]{Proposition}
\newtheorem{lem}[thm]{Lemme}
\newtheorem{cor}[thm]{Corollaire}
\newtheorem{fact}[thm]{Fait}
\def\conclusion{\par\textbf{Conclusion}~}

%%%%%%%%%%%%%%%%%%%%%%%%%%%%%%%

% !TEX root = _global.tex

%%%%%%%%%%%%%%%%%%%%%%%%%%%%%%%%%%%%%%%%%%%%%%%%%%  MACROS


\def\hs{\hbox to 5.5mm{\hfil}}
\def\lb{\vskip.3cm\relax{}}
  
\def\boitelet#1{\hbox to 8mm{\hfil{#1})\hfil}} 
         %hbox pour contenir la lettre designant la reponse 
\def\apreslet{\hskip0.7mm}% espace entre lettre et reponse 
\def\apresrep{\hskip5mm}% espace entre deux reponses sur meme ligne

\def\rA{\noindent\hs\boitelet{a}\apreslet\relax{}}
\def\rB{\apresrep\boitelet{b}\apreslet\relax{}}  
\def\rC{\apresrep\boitelet{c}\apreslet\relax{}}
\def\rD{\apresrep\boitelet{d}\apreslet\relax{}}
\def\rE{\apresrep\boitelet{e}\apreslet\relax{}}

% en passant a la ligne
\def\rrB{\vskip0cm\noindent\hs\boitelet{b}\apreslet\relax{}}   
\def\rrC{\vskip0cm\noindent\hs\boitelet{c}\apreslet\relax{}}  
\def\rrD{\vskip0cm\noindent\hs\boitelet{d}\apreslet\relax{}}  
\def\rrE{\vskip0cm\noindent\hs\boitelet{e}\apreslet\relax{}} 

\usepackage{mathtools} % fixes some bugs in amsmath + provides some useful macros such as \prescript
\mathtoolsset{mathic} % italic correction for maths.
\providecommand{\D}{\mathclose{\,\mathrm{d}}}
\DeclarePairedDelimiter{\intcrochet}{[}{]}
\DeclarePairedDelimiter{\paren}{(}{)}
\DeclarePairedDelimiter{\intercc}{[}{]}
\DeclarePairedDelimiter{\interco}{[}{[}
\DeclarePairedDelimiter{\interoo}{]}{[}
\DeclarePairedDelimiter{\interoc}{]}{]}
\newcommand*{\ens}[1]{\mathbb{#1}} % Ensemble de nombres
\newcommand*{\var}[1]{\mathbf{#1}} % Vari\'et\'e
\newcommand*{\alg}[1]{\mathcal{#1}} % Alg\`ebre
\newcommand*{\RR}{\ens R}%
\newcommand*{\torus}{\ens T}% Tore ! Torus.
\newcommand{\sphere}{\var S}%
\newcommand{\CC}{\ens C}%
\newcommand{\ZZ}{\ens Z}%
\newcommand{\QQ}{\ens Q}%
\newcommand{\NN}{\ens N}%

\PackageInfo{entetes}{Redefining command exp}
\renewcommand{\exp}[1]{{\textup{e}}^{#1}} % On pr\'ef\`ere e^{} que exp{}
\PackageInfo{entetes}{Redefining command vec}
\renewcommand{\vec}[1]{\mathbf{#1}} % D\'esigner un vecteur
\newcommand{\set}[1]{\left\{#1\right\}} % Un ensemble { }
\newcommand*{\abs}[1]{\left\vert#1\right\vert} % Valeur absolue.
\newcommand*{\module}[1]{\left\vert#1\right\vert} % Valeur absolue.
\newcommand*{\norme}[1]{\left\Vert#1\right\Vert} % norme
\newcommand*{\ordre}[1]{\left\vert#1\right\vert} % L'ordre d'un \'el\'ement.
\def\scal(#1,#2){% Produit scalaire.
  \PackageInfo{entetes}{Obsolete command \string\scal}%
  \scalprod{#1}{#2}%
}
\newcommand{\conj}[1]{\overline{#1}}
\newcommand*{\scalprod}[2]{\left\langle #1,#2\right\rangle}
\let\dual\ast

\newcommand{\Donc}{\DOTSB\;\Longrightarrow\;}
\newcommand*{\pardef}{\stackrel{\text{\tiny def}}{=}} % Par
                                % d\'efinition.
\newcommand{\parnot}{\stackrel{\textup{\tiny not}}{=}}
\newcommand*{\iffdefn}{\stackrel{\text{def}}{\iff}} % Par d\'efinition.
\newcommand*{\telque}{\mbox{~t.q.~}} % tel que, dans un ensemble.
\newcommand*{\Defn}[1]{\emph{#1}} %
\newcommand*{\tensorproduct}{\otimes}
\newcommand*{\tensor}{\tensorproduct\PackageWarning{You are using
    deprecated \string\tensor command which conflicts with the /tensor/ package.}}
\newcommand*{\pder}[2]{\frac{\partial #1}{\partial #2}}
\newcommand*{\ddx}[1][x]{\frac{\D}{\D #1}}
%\makeatletter %% \limite[condition]x x_0
\makeatletter
\newcommand*{\limite}[3][]{\lim_{\substack{#2\rightarrow#3\\#1}}}
\makeatother
\def\alldisplaystyle{\everymath{\displaystyle}}
\def\lnot{\neg}
\def\ldonc{\rightarrow}
\def\lssi{\leftrightarrow}
\def\theenumi{\alph{enumi}}

\newcommand{\unitvector}[1]{\vec{e_{#1}}}

\def\UseNoChapterNumbers{%  %% When writing a temporary chapter during the year (nr)
  \renewcommand\thechapter{}
  \renewcommand\thenumexo{\arabic{numexo}}}
\usepackage{preview}
%\PreviewEnvironment{exo}
\PreviewMacro{\thenumexo}
\ifPreview 
\PassOptionsToPackage{hypertex}{hyperref}
\fi
\usepackage{hyperref}
\makeatletter
\ifx\c@Hfootnote\undefined \newcounter{Hfootnote}\fi
\makeatother
% \PreviewMacro{\item}


%%% Local Variables:
%%% TeX-master: t
%%% End:

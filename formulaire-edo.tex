\documentclass[french,oneside,twocolumn,article]{memoir} % {article}% CHERCHER LES FIXMES (occur "FIXME")
\usepackage[utf8]{inputenc} % Permet d'écrire avec les accents
\usepackage[T1]{fontenc}
\usepackage{babel}
\usepackage[margin=1cm]{geometry}
\usepackage{amsthm}
\usepackage[light,frenchstyle]{kpfonts}
\usepackage{xfrac}
\usepackage{mathtools,booktabs}
\usepackage{xargs,xstring}
\usepackage{venndiagram}
\usepackage{amssymb}
\usepackage{amsmath}
\usepackage{amsfonts,calc,enumitem}
\newcommand*{\ddx}[1][x]{\frac{\D}{\D #1}}
\newcommand*{\Defnemph}{\emph}
\newcommand*{\pardef}{\coloneqq} % Par d\'efinition.
\newcommand{\set}[1]{\left\{#1\right\}} % Un ensemble { }
\newcommand*{\abs}[1]{\left\vert#1\right\vert} % Valeur absolue.
\newcommand*{\module}[1]{\left\vert#1\right\vert} % Valeur absolue.
\newcommand*{\norme}[1]{\left\Vert#1\right\Vert} % norme
\newcommand*{\ordre}[1]{\left\vert#1\right\vert} % L'ordre d'un \'el\'ement.
\newcommand{\conj}[1]{\overline{#1}}
\newcommand*{\scalprod}[2]{\left\langle #1,#2\right\rangle}
\newcommand*{\vecprod}{\times}
\newcommand*{\telque}{\mbox{~t.q.~}} % tel que, dans un ensemble.
\newcommand*{\ens}[1]{\mathbb{#1}} % Ensemble de nombres
\newcommand*{\var}[1]{\mathbf{#1}} % Vari\'et\'e
\newcommand*{\alg}[1]{\mathcal{#1}} % Alg\`ebre
\newcommand*{\RR}{\ens R}%
\newcommand*{\TT}{\ens T}% Tore ! Torus.
\newcommand{\sphere}{\var S}% Sph\`ere.
\newcommand{\CC}{\ens C}%
\newcommand{\ZZ}{\ens Z}%
\newcommand{\QQ}{\ens Q}%
\newcommand{\NN}{\ens N}%
\DeclareMathOperator{\im}{Im}
\DeclareMathOperator{\Id}{Id}
\DeclarePairedDelimiter{\intercc}{[}{]}
\DeclarePairedDelimiter{\interco}{[}{[}
\DeclarePairedDelimiter{\interoo}{]}{[}
\DeclarePairedDelimiter{\interoc}{]}{]}
\newcommand*{\limite}[3][]{\lim_{\substack{#2\rightarrow#3\\#1}}}
\newcommand*{\petito}[1]{\mathrm{o}(#1)}
\DeclarePairedDelimiter{\paren}{(}{)}
\DeclarePairedDelimiter{\braces}{\{}{\}}
\DeclarePairedDelimiter{\sqbracket}{[}{]}
\DeclareMathOperator{\pgcd}{pgcd}
\DeclareMathOperator{\ppcm}{ppcm}

\setlist{label=$\ast$} % from enumitem

% \settypeblocksize{*}{450pt}{1.5}
% \setlrmargins{*}{*}{2}
% \setlength{\marginparwidth}{\foremargin-2em}
% \setulmargins{*}{*}{1.3}
% \setlength{\headheight}{15.1pt}
% \checkandfixthelayout% apply the settings.
\def\methode{\paragraph{Méthode}}
\begin{document}
\begin{center}
  Équations différentielles. Math-F-112\\
  Aide-mémoire ;  2015--2016
\end{center}

\paragraph{Remarques}
Il n'est pas permis aux étudiants d'amener leur copie de ce formulaire à l'examen. Néanmoins, les parties pertinentes de ce formulaire \textbf{seront fournies} à l'examen si des équations différentielles font parties du questionnaire.
\chapter{Équations à variables séparées}
\begin{equation*}
  g(y) y' = f(x).
\end{equation*}

\methode{} Intégrer chaque membre : $\int g(y)\D y = \int f(x) \D x$.

\chapter{Équation type homogène}
\begin{equation*}
  y' = f(x,y) \text{ où $f(tx,ty) = f(x,y)$ pour tout $t \in \RR$.}
\end{equation*}

\methode{} Changement de fonction inconnue : \(y(x) = x u(x)\).

\chapter{Équations linéaires premier ordre}
\begin{equation*}
  q(x) y' +  p(x) y = f(x) \tag{EL}.
\end{equation*}

\begin{enumerate}
\item Résoudre l'équation homogène associée (solution générale $y_{SGEH}$).
\item Trouver une solution particulière de l'équation de départ ($y_{SPEL}$).
\item Solution générale : $y(x) = y_{SGEH}(x) + y_{SPEL}(x)$.
\end{enumerate}
\chapter{Équations linéaires second ordre homogène}
\label{second-ordre-lin-homogene}
\begin{equation*}
  a y'' + by' + cy = 0 \qquad a,b,c\in \RR
\end{equation*}

\methode{} Selon les racines du polynômes caractéristique :
\begin{itemize}
\item Deux racines $\lambda_{1}$ et $\lambda_{2}$
  \begin{equation*}
    y(x) = C_{1} \exp{\paren*{\lambda_{1}x}} + C_{2} \exp{\paren*{\lambda_{2}x}}
  \end{equation*}
\item Une racine réelle double, $\lambda$
  \begin{equation*}
    y(x) = (C_{1} + C_{2} x) \exp{\paren*{\lambda x}}
  \end{equation*}
\item Aucune racine réelle, on définit \(\rho = \frac{-b}{2a}\) et \(\omega = \frac{1}{2a}\sqrt{4ac-b^{2}}\)
  \begin{equation*}
    y(x) = (C_{1} \cos(\omega x) + C_{2}\sin(\omega x)) \exp{\paren*{\rho x}}
  \end{equation*}
\end{itemize}
où $C_{1}$ et $C_{2}$ sont les constantes d'intégration.
\chapter{Équations linéaires second ordre non-homogène}
\begin{equation*}
  a y'' + by' + cy = f(x)
\end{equation*}
\begin{enumerate}
\item Résoudre l'équation linéaire homogène associée (solution générale $y_{SGEH}$);
\item Solution particulière $y_{SPEL}$ (cf. infra)
\item Solution générale : $y(x) = y_{SGEH}(x) + y_{SPEL}(x)$
\end{enumerate}

\section{Solution particulière -- Divination}
Selon la forme du second membre $f$, on essaye la forme donnée ci-dessous comme candidat \og solution particulière\fg{} :
\begin{itemize}
\item Si $f$ est un polynôme de degré $n$, essayer un polynôme de degré:
  \begin{itemize}
  \item $n$ si $c \neq 0$
  \item $n+1$ si $c = 0$ et $b \neq 0$
  \item $n+2$ si $b = c = 0$.
  \end{itemize}
\item Si $f(x) = l \textup{e}^{kx}$, essayer :
  \begin{itemize}
  \item $s \textup{e}^{kx}$ si $k$ n'est pas racine du polynôme caractéristique ;
  \item $s x \textup{e}^{kx}$ si $k$ est racine simple ;
  \item $s x^{2} \textup{e}^{kx}$ si $k$ est racine double.
  \end{itemize}
\item Si $f(x) = (l \cos(\tau x) + m \sin (\tau x)) \exp(k x)$, essayer
  \begin{itemize}
  \item $(s \cos(\tau x) + t \sin (\tau x)) \exp(k x)$ si $k + i \tau \in \CC$ n'est pas racine de l'équation caractéristique

    % \paragraph{Remarque}
    % Voici une formulation équivalente, sans nombres complexes : le polynôme caractéristique n'a pas de racine réelle, et $k \neq \rho$ ou $\tau \neq \omega$ ; où $\rho$ et $\omega$ sont les nombres définis dans la méthode du point \ref{second-ordre-lin-homogene}
    
  \item $(s \cos(\tau x) + t \sin (\tau x)) x \exp(k x)$ si il l'est.
  \end{itemize}
\end{itemize}
où $k,l,m,\tau$ désignent des constantes données dans l'équation, et $s,t$ sont des constantes à déterminer.

Si $f$ est une somme de plusieurs termes de la forme ci-dessus, on détermine un candidat pour chaque terme, et on cherche une solution qui sera la somme des candidats.

\section{Solution particulière -- Variation des constantes}
Si $C_{1} y_{1}(x) + C_{2} y_{2}(x)$ est la solution générale de l'équation homogène associée, on cherche une solution particulière sous la forme
\begin{equation*}
  y(x) = u(x) y_{1}(x) + v(x) y_{2}(x)
\end{equation*}
vérifiant de plus
\begin{equation*}
 0 = u'(x) y_{1}(x) + v'(x) y_{2}(x)
\end{equation*}
\end{document}



%%% Local Variables:
%%% TeX-master: t
%%% End:
